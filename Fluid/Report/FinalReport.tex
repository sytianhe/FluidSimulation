%%% template.tex
%%%
%%% This LaTeX source document can be used as the basis for your technical
%%% paper or abstract. Intentionally stripped of annotation, the parameters
%%% and commands should be adjusted for your particular paper � title, 
%%% author, article DOI, etc.
%%% The accompanying ``template.annotated.tex'' provides copious annotation
%%% for the commands and parameters found in the source document. (The code
%%% is identical in ``template.tex'' and ``template.annotated.tex.'')

\documentclass[annual]{acmsiggraph}
\usepackage{amsfonts}
\usepackage{algorithm}
\usepackage{algorithmic}
\TOGonlineid{45678}
\TOGvolume{0}
\TOGnumber{0}
\TOGarticleDOI{1111111.2222222}
\TOGprojectURL{}
\TOGvideoURL{}
\TOGdataURL{}
\TOGcodeURL{}

\title{CS5643 Final Project Report: \\ Modeling Seed Dispersal via Fluid Simulation with Rigid Body Coupling}

\author{Michael Flashman\thanks{e-mail:mtf53@cornell.edu}\\Cornell University \and Tianhe Zhang \thanks{e-mail:tz249@cornell.edu}\\Cornell University}

\pdfauthor{Michael Flashman, Tianhe Zhang}

\keywords{simulation, fluids,  rigid body, plant seeds, dispersion}

\begin{document}

\maketitle

\begin{abstract}
Wind dispersal of  seeds is an important mechanism for mobility in many plant species.  Unlike other dispersal mechanisms, wind dispersal is chiefly a function of seed morphology.  Primarily isolated from the  complex ecology of the plant's environment during this critical stage of a life, physical simulation provides a way to  quantify the fitness of different seed morphologies .   As a first step toward this quantification, we  implement a  framework for stable 2D fluid simulation with rigid body coupling.
\end{abstract}

\keywordlist

\copyrightspace

\section{Overview}


\section{Technical Description}
\subsection{Fluid (Smoke)}
We used the smoke based code for the to simulate the fluid. Followed the architecture outlined in \cite{fattal2004}, we implemented the driving forces and gathering forces to make the smoke function. For the gathering force, we implemented the equation proposed in the paper:
$$ \mathbf{G}(\mathbf{\rho}, \mathbf{\rho}^*)  = \nabla \cdot [\mathbf{\rho} \tilde{\mathbf{\rho}}^* \nabla(\mathbf{\rho} - \mathbf{\rho}^*)]$$
We also used MAC grid when we calculate the gradient.

One of the problem of this method is that the smoke does not preserve. To fix this, we implemented the \texttt{normalizeDensity()} function, which maintains the density sum equal to the sum of keyframe density.

\subsection{Rigid Body Coupling}
We applied the idea presented in \cite{carlson2004} to update the velocity for the rigid body. According to equation (17) in the paper, the forces that arises from the relative density can be expressed as 
$$ \mathbf{S} = \rho_r \mathbf{A}_c + \mathbf{r}_i \times \rho_r \alpha_c - (\rho_r - \rho_f) [\frac{\mathbf{u}^* - \mathbf{u}^n}{\Delta t} + (\mathbf{u}^* \cdot \nabla )\mathbf{u}^* - \mathbf{f}]$$
We decide to not add the collision forces into our system for this project, so our $\mathbf{S}$ term becomes
$$ \mathbf{S}' = - (\rho_r - \rho_f) [\frac{\mathbf{u}^* - \mathbf{u}^n}{\Delta t} + (\mathbf{u}^* \cdot \nabla )\mathbf{u}^* - \mathbf{f}]$$
where $\mathbf{u}^*$ is the velocity of the current iteration of fluid solving. $\mathbf{u}^n$ is the velocity of last iteration. $\rho_r$ is the density of the rigid body, $\rho_f$ is the density of the fluid. $\mathbf{f}$ is the external force.

In order to compute the term $(\mathbf{u}^* \cdot \nabla )\mathbf{u}^*$, assume $\mathbf{u}^* = \binom{u}{v}$, then
\begin{equation}
\begin{split}
(\mathbf{u}^* \cdot \nabla )\mathbf{u}^*& = (u \partial_u + v \partial_v)\mathbf{u}^* \\
& = \binom {u \partial_u u + v \partial_v u}{ u \partial_u v+ v \partial_v v}
\end{split}
\end{equation}
Then we applied the MAC grid to calculate $\partial_u u, \partial_u v, \partial_v u \mbox{ and } \partial_v v$.

Using $\mathbf{S}$ we update $\mathbf{u}^*$ for a new velocity field,
$$\hat{\mathbf{u}} = \mathbf{u}^* + \mathit{w}\frac{\Delta t}{\rho_r}\mathbf{S}$$
where $\mathit{w}$ is the fraction of the cell that is part of solid. To obtain this $\mathit{w}$, we used the supersampling technique. 

In order to maintain the rigidity, we need to obtain $\hat{\mathbf{v}}_i$ and $\hat{\omega}_i$ for each rigid body. We calculate them by integrating $\hat{\mathbf{u}}$ inside a given rigid body $\mathbb{R}_i$ with the following equations:
$$ M_i \hat{\mathbf{v}}_i = \int_{\mathbb{R}_i} \rho_i \hat{\mathbf{u}} dg_i$$
$$ \mathbf{I}_i \hat{\omega}_i = \int_{\mathbb{R}_i} \mathbf{r}_i \times \rho_i \hat{\mathbf{u}} dg_i$$
where, $dg_i$ is the column of the cell occupied by the solid (can be calculated by $\mathit{w}$).
Then, we can find $\hat{\mathbf{u}}_R$ by
$$ \hat{\mathbf{u}}_R = \cup_i (\hat{\mathbf{v}}_i + \hat{\omega}_i \times \mathbf{r}_i) $$
and get our final velocity:
$$\mathbf{u}^{n+1} = (1-\mathit{w})\hat{\mathbf{u}} + \mathit{w}\hat{\mathbf{u}}_R$$
 
In our implementation, we go through each cell in the grid two times. The first time, we find out $\hat{\mathbf{v}}_i$ and $\hat{\omega}_i$ by cumulating $\rho_i \hat{\mathbf{u}} dg_i$ and $\mathbf{r}_i \times \rho_i \hat{\mathbf{u}} dg_i$ at each cell. The second time, we use the calculated $\hat{\mathbf{v}}_i$ and $\hat{\omega}_i$ to compute $ \hat{\mathbf{u}}_R $ and to update the new velocity $\mathbf{u}^{n+1}$ .

\subsection{Stable Solver}
To improve the performance of the fluid solver, we implemented the Pre-conditioner Conjugate Gradient (PCG) solver. The pseudo code is showed in algorithm $\ref{alg1}$.
\begin{algorithm}
\caption{Preconditioned Conjugate gradient algorithm }
\label{alg1}
\begin{algorithmic}
\STATE \textbf{input} $x$, $A$, $b$, $M$, $Q$, $\delta$, $\varepsilon$
\STATE $r\leftarrow b - Ax$
\STATE solve $Qz = r$ for $z$
\STATE $v \leftarrow z$
\STATE $c \leftarrow \langle z, r \rangle$
\FOR{ $k =1$ \textbf{to} $M$ do}
	\IF {$\langle v, v \rangle^{1/2} \le \delta$}
		\STATE exit loop
	\ENDIF
	\STATE $z \leftarrow Av$
	\STATE $t \leftarrow c/\langle v, z \rangle$
	\STATE $x \leftarrow x + tv$
	\STATE $r \leftarrow r - tz$
	\STATE solve $Qz = r$ for $z$
	\STATE $d \leftarrow \langle z, r \rangle$
	\IF {$d \le \varepsilon$}
		\IF {$\langle r, r \rangle \le \varepsilon$}
			\STATE exit loop
		\ENDIF
	\ENDIF
	\STATE $ v \leftarrow z + (d/c)v $
	\STATE $ c \leftarrow d$
	\STATE \textbf{output} k, x, r
\ENDFOR
\end{algorithmic}
\end{algorithm}

\subsection{Polarized Shape Rigid Body}
To make our rigid body look more interesting, we implemented this polar parameterization shape. Given a function whose radius depends on angle, we can generate some different shapes. 

Poler coordinates is convenient to find volume, moment of inertia and the coefficient $\mathit{w}$.

\section{Goal}


\section{Conclusion}
We have described a simulation based approach  for quantifying the aerodynamics of random 2D shapes.  Applied to plant seeds, this approach provides for an (approximate) global understanding of the fitness landscape for seed morphology.  While we only present this project as a proof of concept, our approach has direct implications for the understanding of seed morphology, dispersal, population dynamics, and evolution.
\nocite{*}
\bibliographystyle{acmsiggraph}
\bibliography{bibliography}

\end{document}
