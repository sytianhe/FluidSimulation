%%% template.tex
%%%
%%% This LaTeX source document can be used as the basis for your technical
%%% paper or abstract. Intentionally stripped of annotation, the parameters
%%% and commands should be adjusted for your particular paper � title, 
%%% author, article DOI, etc.
%%% The accompanying ``template.annotated.tex'' provides copious annotation
%%% for the commands and parameters found in the source document. (The code
%%% is identical in ``template.tex'' and ``template.annotated.tex.'')

\documentclass[annual]{acmsiggraph}

\TOGonlineid{45678}
\TOGvolume{0}
\TOGnumber{0}
\TOGarticleDOI{1111111.2222222}
\TOGprojectURL{}
\TOGvideoURL{}
\TOGdataURL{}
\TOGcodeURL{}

\title{CS5643 Final Project Report: \\ Modeling Seed Dispersal via Fluid Simulation with Rigid Body Coupling}

\author{Michael Flashman\thanks{e-mail:mtf53@cornell.edu}\\Cornell University \and Tianhe Zhang \thanks{e-mail:tz249@cornell.edu}\\Cornell University}

\pdfauthor{Michael Flashman, Tianhe Zhang}

\keywords{simulation, fluids,  rigid body, plant seeds, dispersion}

\begin{document}

\maketitle

\begin{abstract}
Wind dispersal of  seeds is an important mechanism for mobility in many plant species.  Unlike other dispersal mechanisms, wind dispersal is chiefly a function of seed morphology.  Primarily isolated from the  complex ecology of the plant's environment during this critical stage of a life, physical simulation provides a way to  quantify the fitness of different seed morphologies .   As a first step toward this quantification, we  implement a  framework for stable 2D fluid simulation with rigid body coupling.
\end{abstract}

\keywordlist

\copyrightspace

\section{Overview}


\section{Technical Description}


\section{Goal}


\section{Conclusion}
We have described a simulation based approach  for quantifying the aerodynamics of random 2D shapes.  Applied to plant seeds, this approach provides for an (approximate) global understanding of the fitness landscape for seed morphology.  While we only present this project as a proof of concept, our approach has direct implications for the understanding of seed morphology, dispersal, population dynamics, and evolution.

\bibliographystyle{acmsiggraph}
\bibliography{bibliography}

\end{document}
