\documentclass[11pt]{article}
%
\include{macros}
\usepackage{graphicx}
\usepackage{pxfonts}
\usepackage{algorithm}
\usepackage{algorithmic}
\usepackage{epstopdf}
\usepackage{fullpage}
\usepackage{epic}
\usepackage{eepic}

\begin{document}

\title{CS5643 Final Project Proposal \\ Fluid Simulation}
\author{Michael Flashman \\ mtf53@cornell.edu \and Tianhe Zhang \\ tz249@cornell.edu\\}
\date{\today}
\maketitle

\abstract{
Wind dispersal of  seeds is an important mechanism for mobility in many plant species.  Unlike other dispersal mechanisms, wind dispersal is chiefly a function of seed morphology (and wind patterns).  Primarily isolated from the  complex ecology of the plant's environment during this critical stage of a life, physical simulation provides a way to  quantify the fitness of different seed morphologies.   As a first step toward this quantification, we will implement a basic framework for stable 2D fluid simulation with rigid body coupling. }


\section{Overview}
%Wind dispersal of  seeds is an important mechanism for mobility in many plant species.  Unlike other dispersal mechanisms, wind dispersal is chiefly a function of seed morphology (and wind patterns).  Primarily isolated from the  complex ecology of the plant's environment during this critical stage of a life, physical simulation provides a way to  quantify the fitness of different seed morphologies. 




Wind dispersal of  seeds is an important mechanism for mobility in many plant species.  Unlike other dispersal mechanisms, wind dispersal is chiefly a function of seed morphology (and wind patterns).  Primarily isolated from the  complex ecology of the plant's environment during this critical stage of a life, physical simulation provides a way to  quantify the fitness of different seed morphologies.   

Several 


Specifically, by sampling from the space of possible seed morphologies and simulating their flight characteristics, it is possible to generate a complete fitness landscape.  From this fitness landscape we may quantify local and global stability of real seeds, or consider morphological pathways in this space.  Our simulation intensive approach has direct implications on the understanding of seed development and plant mobility.  It yields a physically based model for certain evolutionary features in plants.   Our approach yields a quantitative framework in which to consider many theoretical questions from evolutionary, developmental, and ecological biology.  

Detailed numerical analysis 




\section{Technical Components}



\begin{thebibliography}{11}


\end{thebibliography}
\end{document}
