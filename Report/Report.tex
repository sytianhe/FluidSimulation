%%% template.tex
%%%
%%% This LaTeX source document can be used as the basis for your technical
%%% paper or abstract. Intentionally stripped of annotation, the parameters
%%% and commands should be adjusted for your particular paper - title, 
%%% author, article DOI, etc.
%%% The accompanying ``template.annotated.tex'' provides copious annotation
%%% for the commands and parameters found in the source document. (The code
%%% is identical in ``template.tex'' and ``template.annotated.tex.'')

\documentclass[annual]{acmsiggraph}

\TOGonlineid{45678}
\TOGvolume{0}
\TOGnumber{0}
\TOGarticleDOI{1111111.2222222}
\TOGprojectURL{}
\TOGvideoURL{}
\TOGdataURL{}
\TOGcodeURL{}

\title{CS5643 Final Project Proposal: \\ Modeling Seed Dispersal via Fluid Simulation with Rigid Body Coupling}

\author{Michael Flashman\thanks{e-mail:mtf53@cornell.edu}\\Cornell University \and Tianhe Zhang \thanks{e-mail:tz249@cornell.edu}\\Cornell University}

\pdfauthor{Michael Flashman, Tianhe Zhang}

\keywords{simulation, fluids,  rigid body, plant seeds, dispersion}

\begin{document}

\maketitle

\begin{abstract}
Wind dispersal of  seeds is an important mechanism for mobility in many plant species.  Unlike other dispersal mechanisms, wind dispersal is chiefly a function of seed morphology (and wind patterns).  Primarily isolated from the  complex ecology of the plant's environment during this critical stage of a life, physical simulation provides a way to  quantify the fitness of different seed morphologies.   As a first step toward this quantification, we will implement a  framework for stable 2D fluid simulation with rigid body coupling.
\end{abstract}

\keywordlist

\copyrightspace

\section{Introduction}
%Wind dispersal of  seeds is an important mechanism for mobility in many plant species.  Unlike other dispersal mechanisms, wind dispersal is chiefly a function of seed morphology (and wind patterns).  Primarily isolated from the  complex ecology of the plant's environment during this critical stage of a life, physical simulation provides a way to  quantify the fitness of different seed morphologies. 

%Mobility is a central challenges for all living organism.   

Dispersal is the process by which an organism moves way from it's place of birth.  This process is central to understanding  population structure and dynamics, gene-flow, evolution and speciation, as well as many other biological phenomenon\cite{levin1989}.   In general, the mechanism for dispersion is difficult to model precisely. An organism may be mobile for it's entire life, and it's motions may be dictated by complicated behavior and interactions with other species.  For sessile organisms, the dispersal process is often restricted to a single phase in the organism's life in which the organism is essentially passive \cite{nathan2000}.   Such organisms lend themselves to mechanistic understanding.  

Seed plants serve as a model candidate. But even in this passive system, the mechanism for distribution remain surpassingly complex.  Fruits, which beer a seed at it's center, provides one mechanism for seed dispersal, effectively by `convincing' certain frugivores  to eat them. Here, the dispersal process is handed off to an animal which is both much more adept to the task, and much more difficult to model. This plant animal exchange also has non-trivial evolutionary implications; the explosive evolution of fruit bearing plants is  attributed in part to this symbiotic relationships between plant and animals \cite{lorts2008}.  In contrast, wind dispersal is an entirely physical process for dispersion, notably decoupled from the complexity of biology.\footnote{Though wind dispersal is chiefly physical,  seed morphology  plays an important  role during other stages in a plant's life such as during germination. Germination is }

Wind dispersed seeds have been studied previously through empirical analysis of seed morphology and flight characteristics \cite{augspurger1986}.  Much of this work has been applied to  theoretical ballistic and plum models  to obtain plausible explanations for spatial population dynamics and pattern formation \cite{levin2003}. Recent work has also considered  evolutionary implications of spatial segregation of specific plant species as a result of wind dispersal by looking at phenotypic changes across  populations \cite{Cheptou2008}.   A theoretical investigation of dispersal driven speciation is carried out in \cite{levin2010}.  

Taken together, these two approaches to the study of dispersal comprise a closed biological process:  seed morphology determines specific dispersive 
  



, detailed understanding of the affect of phenotypic changeds    

The remains a significant challenge to evolution the effect of phenotypic changes on wind dispersal   


Because wind dispersal of seeds is determined by seed morphology and physics, it is in principle possible to determine 

 connect phenotypic seed traits to  

Taking advantage of the simulation framework developed by 




Wind dispersion of seeds is typically studied as a combination of empirical analysis of real seed flight characteristics and statistical modeling of seed plumes   .  While this approach effectively captures 


 Specifically, by sampling from the space of possible seed morphologies and simulating their flight characteristics, it is possible to generate a complete fitness landscape.  From this fitness landscape we may quantify local and global stability of real seeds.  Or we may consider morphological pathways through this space.  
 
 

Most animals have adapted some physical mobility, making dispersal a process dominated by behavior.  For stationary creatures such as plants, dispersal is  


A detailed understanding of this process is important for understanding many 



\cite{wang2012}
\cite{levin1989}

Several 



\section{Technical Description}
The primary technical challenge of this project is to simulate the dynamics of an arbitrary solid, such as a seed, falling though a fluid, such as air.  This project is only meant to serve as a proof of concept for the research methodology described above, so it will suffice to perform the simulations in 2 dimensions.   

We will implement a stable 2D fluid simulation following the methods outlined in \cite{bridson2006}. This includes MAC grid implementation, boundary conditions implementation and a stable fluid solver. Besides the fluid simulation, we want to add rigid body into our fluid simulation so that fluid can interact with different rigid objects. This will require us to implement the coupling forces described in \cite{carlson2004}. We will also investigate more about the interaction between rigid body and fluid through papers such as \cite{guendelman2005} and \cite{baxter2004}



\section{Goal}

\section{Conclusion}

Our simulation intensive approach has direct implications on the understanding of seed development and plant mobility.  It yields a physically based model for certain evolutionary features in plants.   Our approach yields a quantitative framework in which to consider many theoretical questions from evolutionary, developmental, and ecological biology.  


\bibliographystyle{acmsiggraph}
\bibliography{bibliography}

\end{document}
