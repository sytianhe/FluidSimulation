%%% template.tex
%%%
%%% This LaTeX source document can be used as the basis for your technical
%%% paper or abstract. Intentionally stripped of annotation, the parameters
%%% and commands should be adjusted for your particular paper - title, 
%%% author, article DOI, etc.
%%% The accompanying ``template.annotated.tex'' provides copious annotation
%%% for the commands and parameters found in the source document. (The code
%%% is identical in ``template.tex'' and ``template.annotated.tex.'')

\documentclass[annual]{acmsiggraph}

\TOGonlineid{45678}
\TOGvolume{0}
\TOGnumber{0}
\TOGarticleDOI{1111111.2222222}
\TOGprojectURL{}
\TOGvideoURL{}
\TOGdataURL{}
\TOGcodeURL{}

\title{CS5643 Final Project Proposal: \\ Modeling Seed Dispersal via Fluid Simulation with Rigid Body Coupling}

\author{Michael Flashman\thanks{e-mail:mtf53@cornell.edu}\\Cornell University \and Tianhe Zhang \thanks{e-mail:tz249@cornell.edu}\\Cornell University}
\pdfauthor{Michael Flashman \and Tianhe Zhang}

\keywords{simulation, fluids,  rigid body, plant seeds, dispersion}

\begin{document}

\maketitle

\begin{abstract}
Wind dispersal of  seeds is an important mechanism for mobility in many plant species.  Unlike other dispersal mechanisms, wind dispersal is chiefly a function of seed morphology (and wind patterns).  Primarily isolated from the  complex ecology of the plant's environment during this critical stage of a life, physical simulation provides a way to  quantify the fitness of different seed morphologies.   As a first step toward this quantification, we will implement a  framework for stable 2D fluid simulation with rigid body coupling.
\end{abstract}





\keywordlist

\copyrightspace

\section{Introduction}
%Wind dispersal of  seeds is an important mechanism for mobility in many plant species.  Unlike other dispersal mechanisms, wind dispersal is chiefly a function of seed morphology (and wind patterns).  Primarily isolated from the  complex ecology of the plant's environment during this critical stage of a life, physical simulation provides a way to  quantify the fitness of different seed morphologies. 

Dispersal is the process by which an organism moves away from it's place of birth.    


For most animals, this mechanism is dictated by complex behavior.  For stationary creatures such as plants, and some animals, 

 walking or hoping or crawling from that spot.  

This process is central to understanding many 

Most animals have adapted some physical mobility, making dispersal a process dominated by behavior.  For stationary creatures such as plants, dispersal is  


A detailed understanding of this process is important for understanding many 

\cite{wang2012}


Wind dispersal of  seeds is an important mechanism for mobility in many plant species.  Unlike other dispersal mechanisms, wind dispersal is chiefly a function of seed morphology (and wind patterns).  Primarily isolated from the  complex ecology of the plant's environment during this critical stage of a life, physical simulation provides a way to  quantify the fitness of different seed morphologies.   

Several 


Specifically, by sampling from the space of possible seed morphologies and simulating their flight characteristics, it is possible to generate a complete fitness landscape.  From this fitness landscape we may quantify local and global stability of real seeds, or consider morphological pathways in this space.  Our simulation intensive approach has direct implications on the understanding of seed development and plant mobility.  It yields a physically based model for certain evolutionary features in plants.   Our approach yields a quantitative framework in which to consider many theoretical questions from evolutionary, developmental, and ecological biology.  

Detailed numerical analysis 

\section{Technical Description}

\section{Goal}

\section{Conclusion}

\bibliographystyle{acmsiggraph}
\bibliography{bibliography}

\end{document}
