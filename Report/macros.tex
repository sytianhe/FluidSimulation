\usepackage[parfill]{parskip}
\usepackage{amsmath}   
\usepackage{amsthm}
\usepackage{amssymb} 
\usepackage{calc}
\usepackage{centernot}
\usepackage{ifthen}
\usepackage{graphicx}
\usepackage{enumerate}
\usepackage[mathscr]{eucal}



%MATH FORMATTING
\renewcommand\bf[1]{\ensuremath{\mathbf{#1}}}
\renewcommand\cal[1]{\ensuremath{\mathcal{#1}}}
\newcommand\bb[1]{\ensuremath{\mathbb{#1}}}
\newcommand\call{\stackrel{\text{\tiny call}}{=}}
\newcommand\thru[2]{\ensuremath{#1,\ldots,#2}}
\newcommand\rank[1]{\ensuremath{\text{Rank}(#1)}}
\newcommand\tboxed[1]{\boxed{\text{#1}}}
%\renewcommand\and{\ensuremath{\;\;\;\;\;\;\text{and}\;\;\;\;\;\;} }
\newcommand\where{\ensuremath{\;\;\;\text{where}\;\;\;}}
\newcommand\suchthat{\ensuremath{\;\;\;\text{s.t.}\;\;\;}}
\newcommand\nn{\nonumber}
\newcommand\mcal[1]{\ensuremath{\mathcal{#1}}}
\newcommand\from[1]{\ensuremath{_{#1}}}
\renewcommand\to[1]{\ensuremath{^{#1}}}
\renewcommand\eqref[1]{Eq. (\ref{#1})}
\newcommand\pref[1]{(\ref{#1})}
\newcommand\upa{\ensuremath{\uparrow}}
\newcommand\dna{\ensuremath{\downarrow}}
\renewcommand\deg{\ensuremath{^\circ}}


%GREEK
\newcommand\bs[1]{\boldsymbol{#1}}
\newcommand\Om{\ensuremath{\Omega} }
\newcommand\om{\ensuremath{\omega}}
\newcommand\ep{\epsilon}
\newcommand\g{\gamma}
\newcommand\G{\Gamma}
\newcommand\D{\Delta}
\renewcommand\r{\rho}
\newcommand\m{\mu}
\newcommand\al{\alpha}
\newcommand\la{\ensuremath{\lambda} }


%FUNCTIONS
 \newcommand\csch{\text{csch}}
 \newcommand\inv[1]{\ensuremath{#1^{-1}}}
 \newcommand\vol[1]{\text{vol}(#1)} 
 \newcommand\ceiling[1]{\ensuremath{\left  \lceil #1 \right \rceil}}
 \newcommand\ceil[1]{\ensuremath{\left  \lceil #1 \right \rceil}}
 \newcommand\floor[1]{\ensuremath{ \left \lfloor #1 \right \rfloor}}
  \newcommand\sign[1]{\ensuremath{\text{sign}(#1)}}
 
 %LOGIC
 \newcommand\notimplies{\centernot\implies}

%%SET THEORY%%
\newcommand\seq{\ensuremath{\subseteq}}
\newcommand\nin{\ensuremath{\notin}}
\newcommand\union{\ensuremath{\cup}}
\newcommand\Union{\ensuremath{\bigcup}}
\newcommand\intersection{\ensuremath{\cap}}
\newcommand\intersect{\ensuremath{\cap}}
\newcommand\Intersection{\ensuremath{\bigcap}}
\newcommand\Intersect{\ensuremath{\bigcap}}
\newcommand\es{\ensuremath{\emptyset}}
 
%%ALGEBRA%%
\newcommand\leqc{\trianglelefteq}
 
 %%ANALYSIS
 \newcommand\hs{\ensuremath{\cal{H}} }
 \newcommand\lp[1]{\ensuremath{ {L^#1(\bb{R}^d)}}}
  \newcommand\LpRd{\ensuremath{ {L^P(\bb{R}^d)}}}
  \newcommand\Lp{\ensuremath{ {L^P}}}  
 \newcommand\Rd{\ensuremath{ {\bb{R}^d} }}
 \newcommand\R{\ensuremath{\bb{R}}}
  \newcommand\Q{\ensuremath{\bb{Q}} }
 \newcommand\Z{\ensuremath{\bb{Z}}}
 \newcommand\C{\ensuremath{\bb{C}}}
 \newcommand\N{\ensuremath{\bb{N}}}
 \newcommand\T{\ensuremath{\bb{T}}}
 \newcommand\Td{\ensuremath{ \bb{T}^d}}
 \newcommand\sa{$\sigma$\text{--algebra}}
 \newcommand\order[1]{\ensuremath{\cal{O}(#1)}}

%%PROBABILITY
 \newcommand\sigmafield{\ensuremath{\sigma\text{--field}}}
  \newcommand\lsystem{\ensuremath{\lambda\text{--system}}}
  \newcommand\psystem{\ensuremath{\pi \text{--system}}}
 
 %%PHYSICAL CONSTANTS
 \newcommand\h{\ensuremath{\hbar}}
 \newcommand\hb{\ensuremath{\hbar}}
 
 %%PHYSICS%%
 \newcommand\pb[2]{\ensuremath{ \left [#1, #2 \right ]_{PB}}} 

%% CALC/QUANTUM %%
\newcommand\abs[1]{\ensuremath{\left\vert #1\right \vert}}
\newcommand\Bigabs[1]{\ensuremath{\Bigl\vert #1 \Bigr\vert}}
\newcommand\biggabs[1]{\ensuremath{\biggl\vert #1 \biggr\vert}}

\newcommand\norm[1]{\ensuremath{\lvert \lvert  #1 \rvert \rvert }} 
\newcommand\Tr[1]{\ensuremath{\text{Tr}(#1) }} 
\newcommand\Bignorm[1]{\ensuremath{\Bigl\lvert \Bigl\lvert  #1 \Bigr\rvert \Bigr\rvert }} 

\newcommand\ip[1]{\ensuremath{\left\langle#1\right\rangle}}

\newcommand\avg[2][1]{
	\ifthenelse{\equal{#1}{1}}{\ensuremath{\left \langle#2\right \rangle }}{}
	\ifthenelse{\equal{#1}{2}}{\ensuremath{\left \langle#2^2\right \rangle }}{}
	\ifthenelse{\equal{#1}{3}}{\ensuremath{\left \langle#2\right \rangle ^2}}{}}

\newcommand\derv[3][1]{
	\ifthenelse{\equal{#1}{1}}{\ensuremath{\frac{d#2}{d#3}}}{ 
		\ensuremath{\frac{d^#1 #2}{d#3^#1}}
	}
}

\newcommand\totalderv[2]{\ensuremath{\frac{\text{d}#1}{\text{d}#2}}}

\newcommand\prtl[3][1]{	\ifthenelse{\equal{#1}{1}}{\ensuremath{\frac{\partial#2}{\partial#3}}}{}
	\ifthenelse{\equal{#1}{2}}{\ensuremath{\frac{\partial^2#2}{\partial#3^2}}}{}}
	
\newcommand\db{\ensuremath{\mathchar'26\mkern-12mu d}} 
	
\newcommand\bra[1]{\ensuremath{\langle#1\vert}}
 \newcommand\ket[1]{\ensuremath{\vert #1 \rangle}} 
 \newcommand\bk[2]{\ensuremath{\langle#1\vert#2\rangle}}
\newcommand\re[1]{\ensuremath{\text{Re}(#1)}}
\newcommand\im[1]{\ensuremath{\text{Im}(#1)}}

\newcommand\intall[2][x]{\int_{-\infty}^\infty #2 d #1}	


\newcommand\mathc{\ensuremath{\textbf{C}}}
\newcommand\mathr{\ensuremath{\textbf{R}}}
\newcommand\res{\ensuremath{\text{Res}}}
\newcommand\confeq{\ensuremath{\stackrel{\sim}{\rightarrow}}}	


%MATRIX%
\newcommand\qmatrix[9]{
 \left( \begin{array}{ccc}
 #1&#2&#3\\
 #4&#5&#6\\
 #7&#8&#9
 \end{array} \right)}
 
 \newcommand\sqmatrix[4]{
 \left( \begin{array}{cc}
 #1&#2\\
 #3&#4
  \end{array} \right)}

 \newcommand\matTWO[4]{
 \left( \begin{array}{cc}
 #1&#2\\
 #3&#4
  \end{array} \right)}

 \newcommand\matTHREE[9]{
 \left( \begin{array}{ccc}
 #1&#2&#3\\
 #4&#5&#6\\
 #7&#8&#9
  \end{array} \right)}
 

 \newcommand\mat[2][cccccccccccccccccccccc]{
 \left[ \begin{array}{#1} 
 #2\\
 \end{array} \right]}

% \newcommand\det[2][rrrrrrrrrrrrrrrrrrrrrrrrrrrrrr]{
 %\left| \begin{array}{#1} 
 %#2\\
 %\end{array} \right|}

 \newcommand\qdet[9]{
 \left\vert \begin{array}{ccc}
 #1&#2&#3\\
 #4&#5&#6\\
 #7&#8&#9
 \end{array} \right\vert}

  \newcommand\sqdet[4]{
 \left\vert \begin{array}{cc}
 #1&#2\\
 #3&#4
  \end{array} \right\vert}
 
 \newcommand\qvec[3]{
  \left( \begin{array}{c}
 #1\\
 #2\\
 #3
  \end{array} \right)}

 \newcommand\sqvec[2]{
  \left( \begin{array}{c}
 #1\\
 #2
  \end{array} \right)}
  
  %MATLAB%
  \newcommand\tril[1]{\ensuremath{\text{tril}(#1)}}
  \newcommand\diag[1]{\ensuremath{\text{diag}(#1)}}

  
%DISPLAY  

%ENVIRONMENTS

\theoremstyle{plain}   \newtheorem{thm}{Theorem}
\renewcommand{\qedsymbol}{$\blacksquare$}
\theoremstyle{plain}   \newtheorem{lem}{Lemma}
\theoremstyle{definition}   \newtheorem*{defn}{Definition}
\theoremstyle{plain}   \newtheorem{post}{Postulate}
\theoremstyle{plain}   \newtheorem{prop}{Proposition}
\theoremstyle{plain}   \newtheorem{cor}{Corollary} 
\newcommand\problem[1]{\textbf{Problem #1}\\}
%{\renewcommand{\descriptionlabel}[1]{}

\newcommand\pspace[2][20]{$\hspace{#1pt}$~}